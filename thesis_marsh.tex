%----------------------------------------------------------------------------------------
% PACKAGES AND OTHER DOCUMENT CONFIGURATIONS
%----------------------------------------------------------------------------------------

\documentclass[12pt]{article} % Default font size is 12pt, it can be changed here

\usepackage{graphicx} % Required for including pictures
\usepackage[font={small}]{caption}
\usepackage{subcaption}
\usepackage{float} % Allows putting an [H] in \begin{figure} to specify the exact location of the figure
\usepackage{wrapfig} % Allows in-line images such as the example fish picture

\usepackage{amsmath}
\usepackage{bm}

\usepackage{fancyhdr}

\usepackage{enumerate}
\usepackage[mathscr]{euscript}
\usepackage{listings}
\usepackage{epstopdf}
\usepackage[toc,page]{appendix}
\usepackage{multirow}
\usepackage{hyperref}
\usepackage{bookmark}
\usepackage{booktabs}
\usepackage{dcolumn}
\usepackage{titlesec}
\usepackage{dcolumn}
\usepackage[margin=1in]{geometry}
\usepackage{tikz}
\usepackage{amssymb}
\usetikzlibrary{shapes,shadows,arrows,decorations.markings,calc,spy,backgrounds,patterns,decorations.pathmorphing}
\tikzstyle{model}=[rectangle, rounded corners, thin, draw,align=center]
\usepackage{pgfplots}   
\usepackage{pgfplotstable}
\usepackage{natbib}
\usepackage{makeidx}

\usepackage{pdfpages}

\makeindex

\linespread{1.2} % Line spacing

\graphicspath{{Figures/}} % Specifies the directory where pictures are stored

\numberwithin{equation}{section} % Handles equation numbering


%----------------------------------------------------------------------------------------
% DEFINE TITLE PAGE
%----------------------------------------------------------------------------------------

\makeatletter
\newcommand{\subscript}[1]{\ensuremath{_{\textrm{#1}}}}
\def\s@btitle{\relax}
\def\subtitle#1{\gdef\s@btitle{#1}}
\def\@maketitle{ \linespread{1.0}
  \newpage
  \null
  \vskip 2em%
  \begin{center}%
  \let \footnote \thanks
    {\LARGE \@title \par}%
                \if\s@btitle\relax
                \else\typeout{[subtitle]}%
                        \vskip .5pc
                        \begin{large}%
                                \textsl{\s@btitle}%
                                \par
                        \end{large}%
                \fi
    \vskip 2em%
\includegraphics[scale=0.3]{pc_logo.jpg}
\vskip 2em%
    {\large
      \lineskip .5em%
      \begin{tabular}[t]{c}%
        \@author
      \end{tabular}\par}%
    \vskip 3em%
    {\large \@date}%
  \end{center}%
  \par
  \vskip 1.5em}
\makeatother
\title{\textbf{Contact Binary Stars in Survey Data}}
\author{Franklin Marsh\\
\small{\emph{with advisor}}\\
Philip I. Choi, Ph.D.\\
Professor of Astronomy\\ Pomona College}
\subtitle{A thesis submitted in partial fulfillment of the requirements of a degree of Bachelor of Arts
in\\
Physics\\
at\\
Pomona College}

\begin{document}

%----------------------------------------------------------------------------------------
% TITLE PAGE
%----------------------------------------------------------------------------------------

\maketitle
\thispagestyle{empty}

\newpage

%----------------------------------------------------------------------------------------
% ABSTRACT
%----------------------------------------------------------------------------------------
\thispagestyle{empty}
\begin{abstract}

\end{abstract}
\newpage

%----------------------------------------------------------------------------------------
% ACKNOWLEDGMENTS
%----------------------------------------------------------------------------------------

\thispagestyle{empty}
\section*{Acknowledgments}

To Mom and Dad

\newpage

%----------------------------------------------------------------------------------------
% TABLE OF CONTENTS
%----------------------------------------------------------------------------------------

\tableofcontents % Include a table of contents

\newpage % Begins the essay on a new page instead of on the same page as the table of contents 

%----------------------------------------------------------------------------------------
% INTRODUCTION
%----------------------------------------------------------------------------------------

\section[Introduction - Contact Binaries at the Intersection]{\hyperlink{toc}{Introduction - Contact Binaries at the Intersection}} \label{sec: intro}

%\emph{Q: Why are contact binaries interesting (and important!)?}

The contact binary star is placed at the intersection of some of the largest questions in modern astronomy. In this introduction, we will see how contact binaries connect to a range of issues in modern astrophysics.

Modern observational techniques have allowed for the detection of transients \index{transients}(light sources that appear for a brief time and then disappear) in vast quantities. The supernova \index{supernova} is a common example of a transient. By observing hundreds of supernovae, astronomers discovered that not all supernovae are the same - some are brighter than others, some last longer than others. They have also discovered transients that are not supernovae. In recent years, astronomers have been gaining information about transients that are much brighter than novae, but dimmer than supernovae. They named this class ``Intermediate Luminosity Red Transients". Until recently, there was not a viable physical model for these transients. In late 2008, an ILRT emerged in the constellation of scorpius. When astronomers looked in archival data - they found a contact binary in the spot where the nova had occurred. The leading theory is that the merger of a contact binary system causes these Intermediate Luminosity Red Transients.

While contact binaries systems are very different than the sun, they are important tools for testing the solar-stellar connection\index{solar-stellar connection}: the idea that the sun is similar to other stars and that we can learn about other stars by observing the sun. While the sun takes almost a month to rotate, almost all contact binaries complete a full orbit in less than a day. Contact binaries have much greater angular momenta than single stars of the same spectral type. Contact binaries have strong magnetic fields (as much as 1000 times stronger than the sun's), because they are moving about their rotational axis much more quickly. We will see that a each component of a solar type contact binary exhibits a similar structure to the sun: a radiative inner layer surrounded by a convective envelope. For this reason, contact binaries exhibit the same magnetic phenomena (such as starspots, and flares) as the sun does - except these phenomena on contact binaries are much more dramatic, owing to their stronger magnetic fields. The dramatic magnetic phenomena in contact binaries is observable from large distances. From the earth, we can monitor the magnetic activity of thousands of contact binary stars. This is the subject of much of the original work in this thesis.

Recently, planets have been discovered in orbit around eclipsing binary systems. While planets around contact systems have not been discovered, they are a possibility. Would massive flares and orbital instabilities render life impossible?

With the recent direct observation of gravitational waves by LIGO, there has been renewed interest in gravitational wave sources. It was found that the source of the first gravitational wave detection was two intermediate mass (20 - 30$M_{\odot}$) black holes, which was an unexpected result. Astronomers were uncertain about how two intermediate mass back holes could get close enough to each other to merge. The short-lived, massive contact binary stars offer a solution to this problem. The vast majority of contact binary stars have components with similar masses to the sun. However, a few consist of two very massive O or B type stars. When a O or B type star ends its life, it undergoes a supernova explosion, resulting in a black hole. Some astronomers believe that a star can collapse directly to a black hole, without first undergoing a supernova.

As we will learn, contact binaries are a well-defined class with strict relationships between parameters like mass, luminosity, temperature, and orbital period. This means that by measuring a few parameters, many others can be accurately predicted. There are theoretically and empirically defined relationships between a contact binary's period, temperature, and luminosity. This means, by measuring a contact binary's orbital period (which can be done easily and precisely) astronomers can predict the contact binary's absolute luminosity (which is difficult to measure with traditional methods). For this reason, contact binaries are important \emph{standard candles}\index{standard candle}. Contact binaries are much more common than Cepheid Variables, or RR Lyrae variables. They can be used to trace the structure of the milky way galaxy, and accurately determine distances to other galaxies.

In these ways, the contact binary stands at the intersection of time-domain, solar, gravitational wave, and stellar astronomy. But contact binaries also stand at another important intersection: the intersection of ``old" and ``new" observational techniques. 

We roughly can split observational astronomy into two modes, which I will call:

1. ``Survey Mode": Look out and see what there is to see.

2. ``Target Mode": Observe very specific set of objects in a way tailored to learn about known phenomena.

In the 20th century, the science of astronomy was ``data poor". The limiting factor of discovery was observations from large telescopes of the day. The possession of astronomical data enabled the science. If a scientist had new, proprietary data, science would come out of it. At the turn of the 21st century (enabled by advances in data storage, processing and robotics, and as a direct result of Moore's law) observational astronomical science began to shift modes.

Old telescopes were being remodeled, old gears, motors and lenses were being replaced with robotic systems, enabling their autonomous operation. New telescopes were being constructed with the express purpose of deeply surveying the sky - with minimal human intervention. No longer inhibited by human operators, telescopes could image the sky continuously - dawn to dusk. Data poured from these telescopes like water from a firehose. The new images filled massive stacks of servers: for the first time, astronomers were ``data rich". 

The monstrous stream of data that was provided by these new systems had to be filtered. Scientists interested in new discoveries had to find the proverbial needle in the haystack. The most productive scientist was no longer the scientist with access to the best data, it became the scientist with the best techniques for filtering, stacking, folding, combining, or otherwise analyzing the data. Astronomers started shifting back to ``Survey Mode".

Asteroids were discovered by the thousands. The rate of supernova discovery accelerated from one every few years to approximately \emph{one every night}.  The number of known eclipsing binaries ballooned from just over a thousand, to tens of thousands. The number of galaxies with known distances was increased dramatically by the Sloan Digital Sky Survey. This progress is accelerating: within the decade, at least three major sky surveys of unprecedented depth and cadence will come online.

In the 21st century, we can study thousands of contact binary systems at once, using data from all-sky surveys. This approach presents huge advantages over taking painstaking observations of single contact binary systems. Due to the sheer number of systems studied, conclusions about contact binary behavior can be supported by robust statistics.

However, there are also weaknesses to this approach. Many of the techniques that have been developed for extracting physical information out of observational data do not work well with survey data. We are forced to develop new techniques, and ask different questions.

In \S \ref{sec: background} I provide a brief history of the discovery of the first contact binary star, and outline major leaps of understanding in the field. In \S \ref{sec: observations}, I discuss the types of observations that can be used to learn about contact systems. In \S \ref{sec: analysis_techniques}, I explain the ways that astronomers are able to use measurements to physically characterize contact systems.

I then present original research that I have undertaken with Dr. Tom Prince, Dr. Ashish Mahabal, Dr. Eric Bellm, and Dr. Andrew Drake at the California Institute of Technology. In \S \ref{sec: lc_morph}, we discuss how light-curve characteristics vary as a function of temperature. In \S \ref{sec: dec_var}, we present the results of the search for variability in contact binary luminosity on decadal time scales. In \S \ref{sec: flares}, we present the results of a search for flares on contact binary stars in survey data. Each section is prefaced with a list of driving questions (\emph{Q:}), which are questions that the reader will find answered in that section.

What I hope this thesis is:
1. An introduction to the field of 
2. People have developed techniques for studying the sky when astronomy was data-poor: how can we adapt these techniques to be useful in data-rich astronomy.
3. A roadmap for a promising summer student to use when continuing this work, either at Caltech or Pomona.

\section[The Contact Binary Star]{\hyperlink{toc}{The Contact Binary Star}} \label{sec: background}

\subsection[\emph{Discovery}]{\hyperlink{toc}{Discovery}}

Q: \emph{How was the first contact binary star discovered?}

To understand the history of the study of contact binaries, we must start at the source: the advent of a precise way of measuring the brightness of a celestial object.  

In 1861, J.K.F. Z\"ollner, developed the first practical photometer. In Z\"ollner's photometer \index{Z\"ollner's photometer}, the image of a real star as focused by a 5" objective lens was compared with the light of an artificial star, produced by a bunsen-like gas burner, in the same field of view \citep{staubermann2000trouble}. The brightness of this artificial star could be adjusted by changing the relative orientation of two prisms, until it matched that of the real star. By recording the relative angle of the prisms when the brightness of the artificial and real star were equal, a photometric measurement could be obtained. In the 1860s, Z\"ollner supplied 22 photometers to the great observatories throughout the western world. One of these photometers arrived at the Potsdam Observatory\index{Potsdam Observatory}, 15 miles southwest of Berlin's city center \citep{krisciunas2001brief}.

Karl Hermann Gustav M\"uller \index{M\"uller, Karl Hermann Gustav}, and Paul Friedrich Ferdinand Kempf \index{Kempf, Paul Friedrich Ferdinand} collaborated on observations for the Potsdam \emph{Photometrische Durchmusterung des N\"ordlichen Himmels} (Photometric Catalogue of the Northern Heavens), one of the three great photometric catalogues of the late nineteenth century \citep{bolt2007biographical}. When it was finished, it contained the brightnesses and colors of roughly 14,000 stars down to visual magnitude 7.5 - a monumental undertaking.

While Kempf and M\"uller were making the initial observations for Part III of their \emph{Durchmusterung}, they discovered that two measurements of an otherwise inconspicuous star (the first made in 1899, the second made in 1901) differed by an amount that was greater than was expected. In their survey, each star that showed the potential for variability was observed at a later date to verify the nature of variability.

At the Potsdam Observatory on January 14th, 1903, the sun set at 4:20pm. An hour and a half later, (at 5:56pm) Kempf and M\"uller began constructing a complete light-curve of $BD +56^{\circ}.1400$, which would later be named W Ursae Majoris. They observed until 10:30PM. Follow-up observations three nights later allowed for the construction of the first light-curve of a contact binary star (Figure \ref{fig: muller1903new_1}).

\begin{figure}[H]
\centering
\includegraphics[scale = 0.25]{staubermann2000trouble_4.png}
\caption{Fig. 4 from \citet{staubermann2000trouble}, showing a modern reproduction of a Z\"ollner photometer. Note the tube: the refractor telescope.}
\label{fig: staubermann2000trouble_4}
\end{figure}

\begin{figure}[H]
\centering
\includegraphics[width = \textwidth]{muller1903new_1.png}
\caption{The first light-curve of a contact binary star. Note that the solid curve is interpolated by eye and drawn carefully in pen. Figure 1 from \citet{muller1903new}.}
\label{fig: muller1903new_1}
\end{figure}

The shape of the light-curve was unlike anything that M\"uller and Kempf had seen before, and they struggle to think of a physical system that can produce such a light curve, rejecting many hypotheses, before speculating: \\

\emph{``We may finally consider the hypothesis that the light-variation is produced by two celestial bodies almost equal in size and luminosity whose surfaces are at a slight distance from each other, and which at times almost centrally occult each other in their revolution... On this hypothesis we have only one difficulty, and the not inconsiderable one, as to whether such a system is mechanically possible and can remain stable for any length of time."} \\

This passage marks the beginning of the study of contact binary stars . In this thesis, written 114 years the initial discovery, we will journey to the forefront of contact binary research.

%It turns out that M\"uller and Kempf  were correct in their speculation, as the vast majority of contact binaries are indeed two celestial bodies almost equal in size and luminosity.

%While an increasing number of W UMa systems were discovered in the following years, it took until the late 1960's for a satisfactory physical characterization to be reached. Leon. B. Lucy, at Columbia College \citet{lucy1968light} \citet{lucy1968structure}

%\citet{lucy1968structure} ``It is found that when the common envelope is convective the adiabatic constants of the envelopes of the two companions must be equal."

%``the core solutions of stars with radiative envelopes are insensitive to conditions near the surface...consequently the common envelope will not, in general, be in hydrostatic equilibrium. "

%Even if a physical explanation cannot be obtained \citet{o1951so}

%\citet{eggen1967contact} was the first to discover the period-color correlation. Later, metallicity dependent effects were found by \citet{rubenstein2000metallicity}.

%Based on observations on the Palomar and Mt. Wilson reflectors:

%``It is concluded that contact systems do not form a state in the evolution of detached to semi-detached systems, but spend most of their main sequence life in their present form."

%``The members of cluster and companions in wide visual pairs, plus the available spectroscopic data show that the contact systems have a total luminosity and mass that is equivalent to two equal stars of the observed colour."

\subsection[Physical Characteristics]{\hyperlink{toc}{Physical Characteristics}}

%\emph{Q: What are contact binaries made of? How do contact binaries generate their luminosity? Why are contact binaries shaped like peanuts? How are eclipsing binaries classified? How common are contact binaries compared to all main-sequence stars? How do contact binaries form? How do contact binaries evolve over their lifetime? What is the ultimate fate of a contact binary? What do we know about the most massive contact binaries? What are some interesting magnetic phenomena that occur on contact binaries?  }

In this section, we will work to gain a physical understanding of contact binary systems.

Contact binary stars are made up of two main-sequence stars. In \S \ref{sec: The Main-Sequence Star} we will understand what main-sequence stars are like on the inside, how energy is generated in the cores of main-sequence stars, and how this energy is transported to their surfaces.

Once we have got a firm grasp of the properties of main-sequence stars, we will bring two of them together to form a contact binary. In \S \ref{sec: The Roche Potential}, we learn that we must change the potential that the stellar matter exists in from the point potential to the Roche potential. Also, the components of contact binary stars can transfer mass and energy, from one to the other. We must take this into account when building our model.

In \S \ref{sec: Frequency and Density}, we will learn how common contact binary stars as compared to single main-sequence stars. We will also learn how common they are in the Milky Way galaxy.

In \S \ref{sec: Mechanisms of Formation}, we will learn how contact binaries are formed. We will be introduced to the concepts of angular momentum loss (AML), and Kozai-Lidov cycles.

In \S \ref{sec: Evolution in the Contact State}, we will learn how contact binaries evolve during their lifetimes. We will see how this evolution can drive changes in the observable properties of contact binary systems.

%\citep[p.76, ][]{webbink2003contact} excellent review of remaining problems in contact binary study. 

\begin{figure}[H]
\centering
\includegraphics[scale = 0.5]{lucy1968light_1+garlick.png}
\caption{Model for a contact binary system. The hatched areas denote convection zones, and the vertical dashed line is the axis of rotation. Figure 1 from \citet{lucy1968light}.}
\label{fig: lucy1968light_1}
\end{figure}

\subsubsection[The Main-Sequence Star]{\hyperlink{toc}{The Main-Sequence Star}} \label{sec: The Main-Sequence Star}

In order to understand the internal structure of contact binaries, we must first understand the structure of their two components: main-sequence stars.  The main sequence was an empirically derived group: When astronomers started recording the luminosity and color of lots of stars, they observed that most stars obeyed a relationship between luminosity and color. This relationship can be visualized in an \emph{Hertzprung-Russell Diagram} (or H-R \index{Hertzprung-Russell Diagram} Diagram), like Figure \ref{fig: carroll2006introduction_8_13}. They called the main cluster of points on this diagram the ``Main Sequence". The most familiar example of a main sequence star is our Sun. When a star is fusing hydrogen into helium at its core, we say that it is on the main sequence.

\begin{figure}[H]
\centering
\includegraphics[scale = 0.25]{carroll2006introduction_8_13.png}
\caption{An observer's Hertsprung-Russel (H-R) diagram. The data are from the Hipparcos catalog.Figure 8.13 from \citet{carroll2006introduction}.}
\label{fig: carroll2006introduction_8_13}
\end{figure}

Astronomers have an excellent understanding of the observables (like mass, luminosity, or temperature) of main-sequence stars. Models of main-sequence stars that rely on basic time-independent equations of stellar structure have been successful.

The time-independent equations of stellar structure \index{equations of stellar structure} are a set of relationships between the properties of main sequence stars. They tell how pressure ($P$), enclosed mass ($M_{r}$), enclosed luminosity ($L_{r}$), and temperature ($T$) change as a function of radius $r$. You will notice that that all of the following equations are actually derivatives. When we supply the appropriate boundary condition (eg. ``the temperature $T$ at 1 solar radius is 5800K"), the equations allow for the complete solution of the run of temperature, pressure, and mass through the star.

\begin{equation} \label{stellar_structure1}
\frac{dP}{dr} = -G \frac{M_{r} \rho}{r^{2}} 
\end{equation}

\begin{equation} \label{stellar_structure2}
\frac{dM_{r}}{dr} = 4 \pi r^{2} \rho
\end{equation}

\begin{equation} \label{stellar_structure3}
\frac{dL_{r}}{dr} = 4 \pi r^{2} \rho \epsilon
\end{equation}

\begin{equation} \label{stellar_structure4}
\frac{dT}{dr} = - \frac{3}{4ac} \frac{\bar{\kappa} \rho}{T^{3}} \frac{L_{r}}{4 \pi r^{2}}
\end{equation}

\subsubsection[The Main-Sequence Homology Relations]{\hyperlink{toc}{The Main-Sequence Homology Relations}} \label{sec: The Main-Sequence Homology Relations}

The Main-Sequence Homology Relations (sometimes called the Main-Sequence Scaling Relations) are relationships between the Luminosity $L$, Mass $M$, Radius $R$ ,and temperature $T$ of Zero-age main-sequence\index{Zero-age main-sequence} (ZAMS) stars. 

We can calculate these homology relations using the time-independent equations of stellar structure. \citep{spineda2005homology}

\subsubsection[The Roche Potential]{\hyperlink{toc}{The Roche Potential}} \label{sec: The Roche Potential}

In the equations of stellar structure, there is a hidden assumption. These time-independent equations of stellar structure assume that the stellar matter exists in the potential of a point mass:

\begin{equation} \label{point_mass}
\Psi_{\text{point}} = \frac{GM}{r}
\end{equation}

However, a contact binary system cannot be modeled as a point mass. A contact binary most definitely contains \emph{two} masses, because it contains two stellar components. We can approximate these stellar components as two point masses, separated by a distance $a$.

In contact binary systems, the two stellar components are rapidly rotating about their center of mass.


The Roche model assumes: synchronous rotation, circular orbits, two point masses, in the rotating frame \citep{kopal1959close} :

\citet{mochnacki1984accurate}:

``In Cartesian coordinates, with the origin at the center of mass of the primary, the $x$-axis aligned with the centers of mass, and the $z$-axis parallel to the rotation axis, the potential at a point (x,y,z) co-rotating with binary system is given by: "

\begin{equation} \label{eqn: roche1}
\Psi_{\text{roche}}(x,y,z)= -\frac{G(M_{1} + M_{2})}{2a} C
\end{equation}

where 

\begin{equation} \label{eqn: roche2}
C(x,y,z) = \frac{2}{1+q} \frac{1}{(x^{2} + y^{2} + z^{2})^{\frac{1}{2}}} + \frac{2q}{1 + q} \frac{1}{1 + q[(x -1)^{2} + y^{2} + z^{2}]^{\frac{1}{2}}} + (x - \frac{q}{1 + q})^{2} + y^{2}
\end{equation}

$q = \frac{m_{2}}{m_{1}}$, $(x,y,z)$ are in units of $a$, the separation between the two point masses.

The Roche potential has points where $\nabla \Psi = 0$, called Lagrange Points (see Figure \ref{sluys2006roche}). 

\begin{figure}[H]
\centering
\includegraphics[scale = 0.3]{mochnacki1972model_1.png}
\caption{The coordinate system used in equations \ref{eqn: roche1} and \ref{eqn: roche2} to describe the potential $\Psi$ of a contact binary system. Figure 1 from \citet{mochnacki1972model}}
\label{fig: mochnacki1972model_1}
\end{figure}

\begin{figure}[H]
\centering
\includegraphics[scale = 0.75]{sluys2006roche.png}
\caption{A composite 3D and contour plot of the Roche potential. The Roche lobe is the dark equipotential curve shaped like the $\infty$ symbol. Three out of the five Lagrange points are labelled $L_{1}, L_{2}, L_{3}$.  \citep{sluys2006roche}}
\label{fig: sluys2006roche}
\end{figure}

Now that we understand the shape of the Roche potential, we can learn how the Roche potential is used to classify eclipsing binary stars, in a scheme primarily developed by the work of \citet{kopal1959close}.

In this scheme, eclipsing binaries are classified according to the location of the two photospheres relative to certain Roche equipotentials. In Figure \ref{fig: pringle1985interacting_1_4+terrell2001eclipsing_2+3+4}, we see three types of eclipsing binaries. In Detached systems, the photosphere of each component is well within the Roche lobe (the equipotential curve shaped like $\infty$). In a Semi-detached configuration, the photosphere of one component completely fills its Roche lobe (touching the $L_{1}$ point, while the photosphere of the other component remains well within its Roche lobe. In Overcontact systems, both components \emph{overfill} the Roche lobe, and a bridge of stellar material connects the two components, covering the $L_{1}$ point \citep{terrell2001eclipsing}.

Now we know how Detached, Semi-Detached, and Overcontact binaries are classified - but wait: Where are the \emph{Contact Binaries}? In this section, we have been referring to contact binaries as ``Overcontact Binaries". But why did we have to make the name change?

Since the first half of the 20th century, most astronomers believe that the term ``contact" means that the photospheres of the two components are touching physically. This is incorrect, according to the original classification scheme of \citet{kopal1959close}. He intended contact to mean that the photospheres of the stars were in contact with their Roche Lobes, (the inner Jacobi Equipotential). Though the term "contact" binary is technically incorrect, it is the most common usage in the literature, and is now the accepted name for these kinds of stars. For an excellent review of this naming issue, see \citet{wilson2001binary}

The Roche lobe, (as we have been calling it) is also referred to as the \emph{Inner Jacobi Equipotential} \index{Inner Jacobi Equipotential}. There also exists an \emph{Outer Jacobi Equipotential}. In Figure \ref{fig: pringle1985interacting_1_4+terrell2001eclipsing_2+3+4}, the Outer Jacobi Equipotential is the outer-most curve that is drawn. It passes through the $L_{2}$ point.

An important note on the semantics of binary star classification. \citet{kuiper1941interpretation}

For an excellent review, see: \citep{kallrath2009eclipsing}

\begin{figure}[H]
\centering
\includegraphics[scale = 0.25]{pringle1985interacting_1_4+terrell2001eclipsing_2+3+4.png}
\caption{Types of eclipsing binary systems based on Roche geometry. In the bottom panel, (labelled ``overcontact"), the photosphere of the star (shaded in orange), lies between the inner and outer Jacobi equipotentials. Figures 2,3, and 4 from \citet{terrell2001eclipsing}, and Figure 1.4 from \citet{pringle1985interacting}}
\label{fig: pringle1985interacting_1_4+terrell2001eclipsing_2+3+4}
\end{figure}

\subsubsection[The Geometrical Elements of Contact Binary Systems]{\hyperlink{toc}{The Geometrical Elements of Contact Binary Systems}} \label{sec: The Geometrical Elements of Contact Binary Systems}

In the previous section, we have learned how to differentiate contact binary systems from the other types of eclipsing binary stars. In this section, we will learn how to describe specific contact binaries in terms of their geometrical elements. When we refer to the geometry of the contact binary system, we are really referring to the geometry of its photosphere. The vast majority of what we know of contact binary systems comes from their visible light-curves, which is why there is the convention of treating the photosphere as the ``boundary" of the system.

Astronomers have developed a set of geometrical elements that can describe the location of the photosphere with respect to certain Roche equipotentials \index{Roche equipotential}. 

The first geometrical element of note is the mass ratio $q = \frac{M_{1}}{M_{2}}$\index{mass ratio} of the contact binary. The mass ratio is an important geometrical element in that it influences the shape of the Roche potential. We can see the effect of the mass-ratio term in equation \ref{eqn: roche2}. The Roche potential for a system with a mass ratio of unity (one) is perfectly symmetrical about the $L_{1}$ point. The Roche potential for a system with a mass ratio that is far from unity is not symmetrical about the $L_{1}$ point: the more massive component has an inner Jacobi equipotential that encloses more area.

The second geometrical element is the Roche lobe fill-out factor, $f$\index{fill-out factor}. This is the element that is used to describe the location of the photosphere with respect to the inner and outer Jacobi equipotentials.



\begin{figure}[H]
\centering
\includegraphics[scale = 0.5]{kippenhahn1990stellar_22_7.png}
\caption{The mass values $m$ from centre to surface are plotted against the stellar mass $M$ for zero-age main-sequence models. ``Cloudy" areas indicate the extent of the convective zones inside the models. Two solid lines give the $m$ values at which $r$ is 1/4 and 1/2 of the total radius $R$. The dashed lines show the mass elements inside which $50\%$ and $90\%$ of the total luminosity $L$ are produced. Figure 22.7 from (pp. 212) of \citet{kippenhahn1990stellar}.}
\label{fig: kippenhahn1990stellar_22_7}
\end{figure}

Energy is generated at the core of low mass main sequence stars via the Proton-Proton Chain, or \emph{pp chain}. The pp chain has three branches, each producing helium our of Hydrogen (H), Helium (He) and Beryllium (Be). 

\begin{figure}[H]
\centering
\includegraphics[scale = 0.4]{carroll2006introduction_10_8.png}
\caption{A diagram of pp chain reactions. Percentages by the arrows indicate the branching ratios, revealing that the PP I and PP II chains occur much more frequently than the PP III chain. Figure 10.8 from \citet{carroll2006introduction}.}
\label{fig: carroll2006introduction_10_8}
\end{figure}

The energy produced by all three branches of the pp chain can be represented:

\begin{equation} \label{eqn: pp1}
\epsilon_{pp} = 0.241 \rho X^{2} f_{pp} \psi_{pp} C_{pp} T_{6}^{-2/3} e^{-33.80 T_{6}^{-1/3}} \text{W } \text{kg}^{-1}
\end{equation}

Where $T_{6} \equiv T / 10^{6}$ K. When we expand Equation \ref{eqn: pp1} in a power law about the solar core temperature of $T_{\odot \text{, core}} = 1.5 \times 10^{7}$K, we see that the resulting power law has a $T^{4}$ dependence near $T_{\odot \text{, core}}$:

\begin{equation} \label{eqn: pp2}
\epsilon_{pp} \approx \epsilon_{0, pp}^{'} \rho X^{2} f_{pp} \psi_{pp} C_{pp} T_{6}^{4}
\end{equation}

\subsubsection[The Criterion for Stellar Convection]{\hyperlink{toc}{The Criterion for Stellar Convection}} \label{sec: The Criterion for Stellar Convection}

In our study of contact binary stars, we will find it useful to derive the conditions necessary for stellar convection to occur. We follow closely the analysis presented in \citet{carroll2006introduction}.

The two primary methods of energy transport in main-sequence stars are convection, and radiation. In radiation, the stellar material is in hydrostatic equilibrium, and energy is transported through it via electromagnetic waves. If conditions are such that radiation cannot transport energy away from the core efficiently enough, the stellar material itself will have to move to transport this energy, disrupting hydrostatic equilibrium. We call this disruption convection. 

In their derivation, Carroll and Ostlie envision a bubble of gas with its own temperature, pressure, and density in a surrounding gas medium. 

Absolute magnitudes and luminosities \citep{rucinski1997absolute} \citep{rucinski2006luminosity}

The classical Roche model allows eclipsing binaries to be separated into morphological types \citep{terrell2001eclipsing}.


\subsubsection{Thermal Equilibrium Models}{\hyperlink{toc}{Thermal Equilibrium Models}} \label{sec: Thermal Equilibrium Models}

We know that the two components of a contact binary system have very similar temperatures from observational data. However, observational data also tells us that the two components of contact systems are very rarely equal in mass.

``Up to a third of the energy generated by the primary is transferred to the secondary"\citep{mochnacki1981contact}

\begin{equation} \label{eqn: equilibrium1}
\nabla P = - \rho \nabla \Psi
\end{equation}

\begin{equation} \label{eqn: equilibrium2}
\nabla \rho = \frac{dP(\Psi)}{d\Psi} = \rho(\Psi)
\end{equation}

If we assume a homogenous composition on equipotential surfaces, then all state variables (pressure, density, temperature) are functions of $\Psi$ alone. In order to achieve hydrostatic equilibrium, energy must be exchanged between the two components of the contact binary.

We can construct two model main sequence stars, with masses $M_{1}$ and $M_{2}$. Main sequence stars will obey main-sequence scaling relationships, which are well defined laws the describe how stellar radii, density, and temperature vary with stellar mass \citep{kippenhahn1990stellar}. The mass - radius relationship is particularly important when constructing contact models. 

\begin{equation} \label{eqn: mass_radius}
 \bigg( \frac{R}{R_{\odot}} \bigg) = \bigg( \frac{M}{M_{\odot}} \bigg)^{\alpha}
\end{equation}

\begin{equation} \label{eqn: roche_mass_radius}
\bigg( \frac{R_{1}}{R_{2}} \bigg) = \bigg( \frac{M_{1}}{M_{2}} \bigg)^{0.46}
\end{equation}

Because of the mass - luminosity relationship of main sequence stars, there is a mass ratio - luminosity ratio relationship for the components of contact binaries: 

\begin{equation} \label{eqn: luminosity_radius}
\bigg( \frac{L_{1}}{L_{2}} \bigg) = \bigg( \frac{M_{1}}{M_{2}} \bigg)^{0.9}
\end{equation}


In order for for the binaries to be in contact, their photospheres must touch physically. This allows us to introduce the contact criterion, in which the separation between the centers of the components are equal to the sum of the two radii:

\begin{equation} \label{eqn: contact_criterion}
a = R_{1} + R_{2}
\end{equation}

We can relate to the combined masses of the two stars to their periods using the generalized form of Kepler's Third Law: 

\begin{equation} \label{eqn: kepler3}
P^{2} = \frac{4\pi^{2}}{G(M_{1} + M_{2})} a^{3}
\end{equation}


Advection is a transport mechanism of a substance, or property (e.g. temperature, density) by a fluid due to the fluid's bulk motion.

\citep{shu1976structure} contact discontinuity model

\citep{lubow1977structure}

\citet{gazeas2008angular} have shown that orbital parameters of contact binary systems obey certain relationships.

\begin{equation} \label{gazeas2008angular_6}
P = 0.1159 * a^{\frac{3}{2}} M^{\frac{1}{2}}
\end{equation}

Where $P$ is the orbital period in days, $M = M_{1} + M_{2}$ is the total mass of the binary system, and $a$ is the semi major axis of the system in meters.

We can also calculate the orbital angular momentum $H_{\text{orb}}$ 

\begin{equation} \label{gazeas2008angular_7}
H_{\text{orb}} = 1.25 \times 10^{52} * M^{\frac{5}{3}} P^{\frac{1}{3}} q(1 + q)^{-2}
\end{equation}

where $q = \frac{M_{1}}{M_{2}}$ is the mass ratio of the components.


\citep{gazeas2006masses}

\subsection{Interior Structure}{\hyperlink{toc}{Interior Structure}} \label{sec: Interior Structure}

\begin{figure}[H]
\centering
\includegraphics[scale = 0.25]{lubow1977structure_2.png}
\caption{An equatorial cross-section of a 1 $M_{\odot}$ + 0.5 $M_{\odot}$ zero-age contact binary of solar composition. The filling factor of this model is $f = 0.41$, and the binary period is $P_{d} = 0.228$ days. Fig. 2 from \citet{lubow1977structure}}
\label{fig: lubow1977structure_2}
\end{figure}

\begin{figure}[H]
\centering
\includegraphics[scale = 0.25]{lubow1977structure_3.png}
\caption{An equatorial cross-section of a 2 $M_{\odot}$ + 1 $M_{\odot}$ zero-age contact binary of solar composition. The filling factor of this model is $f = 0.84$, and the binary period is $P_{d} = 0.314$ days. Fig. 3 from \citet{lubow1977structure}}
\label{fig: lubow1977structure_3}
\end{figure}

\subsubsection{Common-Envelope Evolution}{\hyperlink{toc}{Common-Envelope Evolution}} \label{sec: Common-Envelope Evolution}

Common-envelope Evolution, (CEE) \citep{ivanova2013common}

\subsection[Frequency and Density]{\hyperlink{toc}{Frequency and Density}}

 \citep{rucinski1998contact} Studies using OGLE data on the galactic bulge (Baade's Window) indicates that the frequency of contact binaries relative to main sequence stars (or spatial frequency) is approximately $\frac{1}{130} = 0.7\%$, in the absolute magnitude range of $2.5 < M_{v} < 7.5$. A later study using ASAS data shows that the spatial frequency is 0.2\% in the solar neighborhood \citep{rucinski2006luminosity} in the absolute magnitude range of $3.5 < M_{v} < 5.5$.

A catalog of 1022 contact binary systems in ROTSE - 1 data placed the the space density of contact binaries at $1.7 \pm 10^{-5} \text{pc}^{-3}$ \citep{gettel2006catalog}
 
Contact binaries are the most frequently observed type of eclipsing binary star, because their eclipses can be detected at a wide range of orbital inclinations. In recent searches for eclipsing binary stars in survey data \citep{drake2014catalina} contact binary stars comprised 50\% of the new variable stars discovered. 

Existing catalog of contact binaries in the field \citep{pribulla2003catalogue}


\subsection[Mechanisms of Formation]{\hyperlink{toc}{Mechanisms of Formation}}

Formed in Contact. \citet{lucy1968structure} was incorrect in assuming that contact systems exist at Zero Age Main Sequence (ZAMS).

\citep{yildiz2013origin}

\citep{bilir2005kinematics}

\citep{li2007formation}

Initially Detached, but then inspiral
Angular Momentum Loss through stellar winds. 


In recent years, evidence has been amassing for the formation via companion pathway. In this pathway, two stars begin in a stable orbit. When a third star is introduced into the system, it steals angular momentum from the first two stars, resulting in a closer orbit. 

There is evidence that this companion stays in orbit around the contact binary.
A study by \citet{pribulla2006contact} has established a lower limit on the number of triple systems. Angular Momentum Loss through tertiary components.  \citep{lohr2015orbital} In a search of 13,927 eclipsing binaries in the SuperWASP catalog, 24\% had period-changes indicating a closely orbiting companion.

In the Kozai-Lidov mechanism \index{Kozai-Lidov mechanism}, the orbit of two inner bodies is perturbed by a third body orbiting farther out. The equations of motion for the three-body system, a specific angular momentum is conserved:

\begin{equation} \label{kozai_1}
L_{z} = \sqrt{1- e^{2}} \cos i
\end{equation}

Because the quantity $L_{z}$ is conserved, orbital eccentricity $e$ can be traded for orbital inclination $i$. So, three body systems with undergo Kozai-Lidov cycles, with a certain period:

\begin{equation} \label{kozai_2}
T_{\text{Kozai}} = 2 \pi \frac{\sqrt{GM}}{G m_{2}} \frac{a_{2}^{3}}{a^{\frac{3}{2}}} (1 - e^{2}_{2})^{\frac{3}{2}}
\end{equation}


\subsection[Evolution in the Contact State]{\hyperlink{toc}{Evolution in the Contact State}}

Thermal Relaxation Oscillations \citep{wang1994thermal}. May cause orbital period changes observed in \citet{qian2001orbital}.

Kahler's first paper on the structure of contact binaries \citep{kahler2002structure}

The structure of contact binaries \citep{kahler2004structure}

the surface of the contact system does not obey a simple gravity brightening law \citep{kahler2004structure}, \citep{hilditch1988evolutionary}.

\citep{rubenstein2001effect}

Angular momentum and mass evolution. Some of the most recent modeling work indicates that the typical duration of the contact state is 1 to 1.5 Gyr \citep{gazeas2008angular}.

masses and angular momenta \citep{gazeas2006masses} angular momentum loss \citep{vilhu1981contact}

contact binaries occupy a very narrow range of parameter space \citep{gazeas2009physical} \citep{awadalla2005absolute}

Overall evolution \citep{stepien2008evolutionary}, 2016 review of close binary evolution \citep{tutukov2016evolution}

Short period limit \citep{rucinski2007short} \citep{drake2014ultra} \cite{lohr2012period} \citep{rucinski1992can}


\subsection[Evolution out of the Contact State]{\hyperlink{toc}{Evolution out of the Contact State}}

It is generally accepted that the contact state of binary evolution ends with the inspiral and merger of the two components. The merger event is where the contact system becomes dynamically unstable, and rapidly coalesces into a single, rapidly rotating star. 

The tidal interaction between the two components of a contact binary star.

To understand when and why the contact state comes to an end, we must understand the range of conditions over which the contact state is stable.

Merger \citet{tylenda2011v1309}.

\begin{figure}[H]
\centering
\includegraphics[scale = 0.25]{tylenda2011v1309_1.png}
\caption{Light curve of V1309 Sco from the OGLE-III and OGLE-IV projects: $I$ magnitude versus time of observations in Julian Dates. Time in years is marked on top of the figure. At maximum the object attained $I \approx 6.8$. Figure 1 from \citet{tylenda2011v1309}}
\label{fig: tylenda2011v1309_1}
\end{figure}

\begin{figure}[H]
\centering
\includegraphics[scale = 0.25]{tylenda2011v1309_2.png}
\caption{Figure 2 from \citet{tylenda2011v1309}}
\label{fig: tylenda2011v1309_2}
\end{figure}


The contact binary in the OGLE merger had a period of approximately 1.4 days. Long-period contact binaries \citep{rucinski1998eclipsing} 

Blue straggler \citet{andronov2006mergers}

stability of the contact configuration \citep{rasio1995minimum}

minimum mass ratio \citep{arbutina2009possible} \citep{arbutina2009}

One of the ways that a contact binary system can merge is called the Darwin instability. In a Darwin instability, the ...

In the early 1990s, large numbers of new contact binaries were discovered among blue stragglers, in open and globular clusters 

\citep{kaluzny1988ccd} no discovery in six open clusters.

In the globular cluster M71, four contact binaries discovered by \citet{yan1994primordial}, placing a lower limit of $1.3\%$ on the frequency of primordial binaries in M71 with initial orbital periods in the range of 2.5 to 5 days. 

Short period eclipsing binaries have been found among blue stragglers in the globular cluster NGC5466 \citep{mateo1990blue}.

Review of binaries in globular clusters \citep{hut1992binaries}

modern work on six binaries in NGC188 \citep{chen2016physical}

\begin{figure}[H]
\centering
\includegraphics[scale = 0.25]{mateo1990blue_1.png}
\caption{A color-magnitude diagram of globular cluster NGC 5466. The blue stragglers are defined to be all stars located within the region bounded by the dashed lines. The mean $V$ magnitudes and $(B - V)$ colors of the eclipsing binaries discovered by \citet{mateo1990blue} are shown as open circles. Figure 1 from \citet{mateo1990blue}.}
\label{fig: mateo1990blue_1}
\end{figure}

\subsection[Early-Type Contact Binaries]{\hyperlink{to}{Early-Type Contact Binaries}}

When astronomers say that a star is ``Early-Type" \index{Early-Type}, they mean that it is ``early" on the spectral classification sequence (OBAFGKM). O and B stars are more massive, more luminous and hotter than our sun. Early-Type contact binaries deserve a special section due to their extreme mass, luminosity, rarity, and short lifetime. There are only a handful of O and B Type contact binary systems known. The four best studied systems are TU Muscae \citep{penny2008tomographic}, MY Cam \citep{lorenzo2014my}, UW CMa \citep{antokhina2011light}, V382 Cygni \citep{popper1978masses}. 

Over forty have been found in the Small Magellanic Cloud \citep{hilditch2005forty} 

Additional SMC study of intermediate period system \citep{priya2013photometric}.

Massive contact binaries in M31 \citep{lee2014properties}, \citep{vilardell2006eclipsing}

In modern survey data, we can observe individual eclipsing binary systems in nearby galaxies.


\subsection[Magnetic Activity]{\hyperlink{toc}{Magnetic Activity}}

In the solar atmosphere, the movement of plasma in the convective region creates magnetic fields, which in turn affect the motion of that same plasma. Contact binaries are rapidly rotating systems, with orbital periods of 0.2 to 0.8 days (compare this rate with the approximately 30 day solar rotational period), so they have the potential to form much stronger magnetic fields.

There is a lot of evidence that contact binary stars have strong magnetic fields. Astronomers observe changes in their light-curves, indicating that starspots may be appearing and disappearing on their photospheres. Doppler imaging of contact binaries can reveal the shapes and locations of the starspots on the photosphere (Fig. \ref{fig: barnes2004high_5})

 Because late-type contact binaries have Common Convective Envelopes (or CCEs), 

Some authors believe that the existence of the CCE buries the very strong surface magnetic field, which could prevent the production of flares \citep{qian2014optical}. 

\begin{figure}[H]
\centering
\includegraphics[scale = 0.40]{qian2014optical_15.png}
\caption{Fig. 15 from \citet{qian2014optical}}
\label{fig: qian2014optical_15}
\end{figure}

\citep{balogh2015solar} book review of solar magnetic activity cycles.

The applegate mechanism \citep{applegate1992mechanism} \citep{lanza2006internal}

W UMa as X-ray sources \citep{stepien2001rosat}

Observational starspots and magnetic activity cycles.  \citep{borkovits2005indirect,qian2000possible,kaszas1998period,qian2007ad,lee2004period,yang2012deep,zhang2004long}.

It is possible that starspots are responsible for the variation in brightness observed by CRTS. Doppler imaging techniques have confirmed the presence of large starspots on the surface of some contact binaries \citep{barnes2004high}.  An example of a well observed change in the contact binary light-curve \citep{gazeas2006modeling}.

\citet{barnes2004high} has used an Echelle spectrograph on the 3.9-meter Anglo-Australian Telescope to perform doppler imaging of the AE Phe system (P = 0.362 d), revealing that the photosphere of the system is heavily spotted (Fig. \ref{fig: barnes2004high_5}).

\begin{figure}[H]
\centering
\includegraphics[scale = 0.25]{barnes2004high_5.png}
\caption{Fig. 5 from \citet{barnes2004high}}
\label{fig: barnes2004high_5}
\end{figure}

Contact binaries are known X-ray sources \citep{chen2006w}. Flares have been observed in X-ray bands using ROSAT \citep{mcgale1996rosat}.
EXOSAT has been used to observe a flare in X-ray and Microwave data on VW Cephei (P =  0.28 days, $T_{1}, T_{2}$ = 5500K, 5000K) \citep{vilhu1988simultaneous}. Extreme UV observations have identified coronal characteristics \citep{brickhouse1998extreme}. Long time series observations of single m-dwarfs  \citep{lacy1976uv}

During a continuous monitoring campaign in the winters of 2008 and 2010, \citet{qian2014optical} observed a contact binary system CSTAR 038663 ($P = 0.27$ days, $T_{1}, T_{2} =$ 4616K, 4352K) for a total of 4167 hours (174 days) in the SDSS $i$ band using the CSTAR telescope array in the Antarctic. In this time, \citet{qian2014optical} discovered 15 $i$ band flares, revealing a flare rate of $0.0036$ flares per hour. These 15 flares had durations ranging from 0.006 to 0.014 days (9 to 20 minutes), and amplitudes ranging from 0.14 - 0.27 magnitudes above the quiescent magnitude.

In 1049 close binaries observed by Kepler, \citet{gao2016white} have identified 234 ``flare binaries", on which a total of 6818 flares were detected. Kepler's continuous monitoring capability and precise photometry make it extremely well suited to the detection of white-light flares \citep{walkowicz2011white}. While CRTS does not match Kepler's observing cadence, photometric precision, or ability to observe a given target continuously, it observes 33,000 square degrees a much larger area of the sky than Kepler does (100 square degrees) \citep{drake2009first, basri2005kepler}.

M-dwarfs have previously been searched for flares in Sloan Digital Sky Survey Stripe 82 data \citep{kowalski2009m}. Our study will be similar to that of \citet{kowalski2009m}, because we are also using survey observations of large regions of sky, as opposed to the continuous monitoring studies. \citet{hilton2010m} have discovered flares in the time resolved SDSS spectroscopic sample, using a Flare Line Index based on H$\alpha$ and H$\beta$ line strength.

The evolution and migration of starspots on contact binaries has been tracked with doppler imaging \citep{hendry2000doppler} and more recently, in Kepler data \citep{tran2013anticorrelated, balaji2015tracking}. Starspots are magnetic phenomenon, and so their occurrence is related to the magnetic activity of their host star \citep{berdyugina2005starspots}.  

\subsection[Remaining Questions]{\hyperlink{toc}{Remaining Questions}} \label{sec: Remaining Questions}

\section[Observations]{\hyperlink{toc}{Observations}} \label{sec: observations}

\emph{Q: What kind of observations can astronomers obtain from contact binary systems?}

Astronomy is unique as a science because all the information that can be obtained from an object in the sky comes to us as electromagnetic waves. Perhaps \emph{THE} question in observational astronomy is: ``What can we learn from these electromagnetic waves?". The study of contact binary stars is no different. In this section, we will learn the ways that researchers study electromagnetic waves from contact binary stars.

\subsection[Images of Contact Binaries]{\hyperlink{toc}{Images of Contact Binaries}} \label{sec: Images of Contact Binaries}

The oldest type of astronomical information is image data: ``What do I see when I look through the telescope?". To put this question in more formal language: ``What is the distribution of the intensity of visible light as a function of position?". When we look at the moon, for example, we can learn a lot about it: we might see some crater ``over here", with a given size, shape, color, etc. We might see a dark lunar mare (or ``sea"), ``over there", with another size, shape, color, etc. The moon is what we call a ``resolved source", meaning that features on it are distinguishable: we can separate ``over here" from ``over there". In other words, the distance between ``over here" and ``over there" is larger than the resolution limit of our telescope.

Let's see if we can reasonably obtain image data from a contact binary:

On a still, clear night at the Las Campanas observatory in Chile, the atmospheric resolution limit (or ``seeing") is 0.5 arcseconds. This is the best resolution that can be expected from a telescope on earth: Chile's Atacama desert is know for some of the best seeing on earth.

\begin{equation} \label{eqn: arcseconds}
0.5 \text{ arcseconds} * \frac{1}{3600} \frac{\text{arcseconds}}{\text{degrees}} = 1.4 \times 10^{-4} \text{ degrees} * \frac{\pi}{180} \frac{\text{radians}}{\text{degrees}} = 2.4 \times 10^{-6} \text{ radians}
\end{equation}

In order to distinguish between the two components of a contact binary, the resolution limit of our telescope must be smaller than the distance between the centers of the two components. 

For a contact binary star of solar type, this is about one solar radius: $1 R_{\odot} = 6.957 \times 10^{5} \text{ km}$. Let us place this hypothetical contact binary at the same distance as the nearest star, \emph{Proxima Centauri}, which is $4.243 \text{ light years} = 1.301 \text{ parsecs} = 4.014 \times 10^{13} \text{ km}$.

To calculate the angle that a solar type-contact binary at the distance of \emph{Proxima Centauri} would subtend, we will use the small angle approximation:

\begin{equation} \label{eqn: smallangle}
\sin(\theta) \approx \theta, \qquad \cos(\theta) \approx 1 - \frac{\theta^{2}}{2}, \qquad \tan(\theta) = \frac{\sin(\theta)}{\cos(\theta)} = \frac{\theta}{1 - \frac{\theta^{2}}{2}} \Rightarrow \tan(\theta) \approx \theta
\end{equation}

If we set up a right triangle (as in Fig. \ref{fig: triangle}), we see than the tangent of the angle $\theta$ is equal to the radius of the sun divided by the distance to \emph{Proxima Centauri}.

\begin{figure}[H]
\centering
\includegraphics[scale = 0.25]{triangle.png}
\caption{Calculating the angle $\theta$ subtended by a solar-type contact binary at the distance of the nearest star.}
\label{fig: triangle}
\end{figure}

\begin{equation} \label{eqn: example_angle}
\frac{R_{\text{sun}}}{d_{\text{proxima centauri}}} = \frac{6.957 \times 10^{5} \text{ km}}{4.014 \times 10^{13} \text{ km}} = \tan(\theta) \approx \theta = 1.733 \times 10^{-8} \text{ radians}
\end{equation}

When comparing the resolution necessary to distinguish the components of a contact binary to the best resolution possible on earth:

\begin{equation} \label{eqn: resolution_comparison}
\frac{1.733 \times 10^{-8} \text{ radians}}{2.4 \times 10^{-6} \text{ radians}} \approx 0.01
\end{equation}

To summarize: we would need 100 times the resolving power achievable from the earth to obtain image data from a large contact binary at the distance of the nearest star. In actuality, the situation is worse. 44 Bootis is the nearest contact binary system to earth, at a distance of 13 parsecs (42 light years) it is 10 times further away than \emph{Proxima Centauri} \citep{eker2008new}. For this reason, we cannot obtain usable image data from contact binaries \footnote{it is possible to achieve this resolution (as good as 0.0005") through long-baseline interferometry. Using the CHARA array on Mount Wilson, researchers have constructed a resolved image of the eclipsing binary system $\beta$ \emph{Lyrae} \index{$\beta$ Lyrae} \citep{zhao2008first}. However, interferometric imaging is only possible for the brightest stars, so is not useful for contact binaries.}.

\subsection[Photometry of Contact Binaries]{\hyperlink{toc}{Photometry of Contact Binaries}} \label{sec: Photometry of Contact Binaries}

\emph{Q: Why does the flux received from a contact binary vary over time?}
\emph{Q: How was the light-curve used to determine what a contact binary was?}
\emph{Q: How can light-curve data be used to learn about contact binary systems? }

In images, contact binaries appear as an unresolved point source. At first glance, it may appear that astronomers are stuck: they cannot ``see" the contact binary and so must remain uncertain about its characteristics. However, as Kempf and M\"{u}ller learned in 1903, the amount of light received from a contact binary varies as a function of time. This function is called the light-curve: \index{light-curve}

\begin{equation} \label{eqn: light_curve}
f(\text{Time}) = \text{ Flux Received at Telescope}
\end{equation}

A light curve is constructed from observations: by repeatedly measuring the brightness of a source over a certain time span, an astronomer can sample the light-curve and approximate its true shape.

Kempf and M\"{u}ller knew that they could used the light-curve to learn about the shape of the contact binary system. First, they noted that the light-curve was periodic: after a certain amount of time, the trend in flux \emph{exactly repeated} itself. Thus they knew that the process that was responsible for the variation in the flux was cyclical in nature. 

They knew that the period of the light variation in W UMa was very stable (``The error of the period can hardly be more than 0.5s..."). They assumed that a rotational or orbital mechanism was responsible for the light variation. They thought that the presence of a large dark spot on a rapidly rotating single star, which was hypothesized to be ``in an advanced stage of cooling". However, W UMa was a white star, not a cool red star, leading Kempf and M\"{u}ller to discredit this model. They also considered a single star in the shape of an ellipsoid - a large, however they calculated that this model did not describe the shape of the light-curve very well. In 1903, the eclipsing binary model was already proposed as a mechanism for the light-curves of certain stars (most notably Algol \index{Algol}). To construct their model, they looked at existing eclipsing binary light curves and imagined what would happen if they brought the two stars close together. If they brought the two stars close enough together so that the stars were almost touching, there was always variation in the light-curve, just like they observed.

\begin{figure}[H]
\centering
\includegraphics[scale = 0.25]{lc_anatomy.png}
\caption{Light-curve is from CRTS data \citep{drake2014catalina}. Illustration of contact binary phase from an animation at: \url{http://cronodon.com/SpaceTech/BinaryStar.html}}
\label{fig: lc_anatomy}
\end{figure}

The shape of a contact binary's light-curve can tell us a lot about it. Indeed, the aim of much of the original work in this thesis is to determine how the shape of the contact binary light-curve correlates with physical parameters. I'll now go through a few light-curve features 

Not all contact binary light-curves look the same, so why is that?

What information does can be learned about a contact binary system based only on it's light-curve?

This means that the light-curve of a system with two (relatively) small and cool $M$ type components should be qualitatively and quantitatively identical to the light-curve of a system with two massive and hot $G$ type components, given the systems have the same geometry $[f,q,i]$.

\subsubsection[Amplitude]{\hyperlink{toc}{Amplitude}} \label{sec: Amplitude}

One of the most obvious features of the contact binary light-curve is its Amplitude \index{Amplitude, light-curve}. The amplitude is how much the brightness of the contact binary varies from its minimum brightness, to its maximum brightness.

\subsubsection[Difference Between Eclipse Minima]{\hyperlink{toc}{Difference Between Eclipse Minima}} \label{sec: Difference Between Eclipse Minima}

\subsubsection[Difference Between Out-of-Eclipse Maxima]{\hyperlink{toc}{Difference Between Out-of-Eclipse Maxima}} \label{sec: Difference Between Out-of-Eclipse Maxima}

\subsubsection[Eclipse Width at Half-Minimum]{\hyperlink{toc}{Eclipse Width at Half-Minimum}} \label{sec: Eclipse Width at Half-Minimum}

How much can the shape of the light-curve tell us about the geometry of the contact binary system? When astronomers constructed the first computer models of contact binaries (based on the initial solution in \cite{lucy1968light}

\subsection[Spectra of Contact Binaries]{\hyperlink{toc}{Spectra of Contact Binaries}} \label{sec: Spectra of Contact Binaries}

time-series spectra are some of the most complete observations of contact binaries. 

\citep{hrivnak1989radial} radial velocities and IR data of OO Aql.

Analyzing spectra of contact binaries is challenging. Abundances of elements cannot be determined precisely due to the broadening and blending of spectral lines caused by the fast rotation \citep{gazeas2006masses}. 

\subsection[X-ray and Ultraviolet Data on Contact Binaries]{\hyperlink{toc}{X-ray and Ultraviolet Data on Contact Binaries}} \label{sec: X-ray and Ultraviolet Data on Contact Binaries}

Contact binaries are much brighter than main-sequence stars in the ultraviolet, owing to their strong magnetic fields cause by rapid rotation. Earth's atmosphere is opaque to X-ray and ultraviolet (UV) light, but the advent of space-based observatories has made observations of the sky possible in these passbands.

\citep{cruddace1984contact} first \emph{Einstein} survey of 17 contact binaries. Fig 2 has the X-ray luminosity- period relation.
\citep{vilhu1987chromospheric} ``the two components of a contact binary have identical chromospheres and transition regions" $F_{x} / F_{bol}$ remains roughly constant with $(B - V)$ color.
\citep{stepien2001rosat} supersaturation in contact binaries.


In this section, I describe some of the major sources of modern observations of contact binaries. 

Examples of such surveys are the All-Sky Automated Survey \citep[ASAS,][]{pojmanski2000all}, Robotic Optical Transient Search Experiment \citep[ROTSE,][]{akerlof2000rotse}, Trans-Atlantic Exoplanet Survey \citep[TrES,][]{devor2008identification}, Lincoln Near-Earth Asteroid Research program \citep[LINEAR,][]{palaversa2013exploring}, and Catalina Real-Time Transient Survey \citep[CRTS,][]{drake2014catalina}. Researchers have also selected pure samples of contact binary systems from large survey data sets for study. Researchers have previously used data from the Optical Gravitational Lensing Experiment \citep[OGLE,][]{rucinski1996eclipsing}, Super Wide Angle Search for Planets \citep[SuperWASP,][]{norton2011short}, and CRTS \citep{drake2014ultra} to construct pure contact binary samples for study. \citet{lee2015properties} have used this approach to study a pure sample detached eclipsing binaries from the CRTS variable catalog. 

\begin{figure}[H]
\centering
\includegraphics[scale = 0.25]{modern_surveys.png}
\caption{Images of the instruments used in six modern surveys. In the top left, SuperWASP \citep[SuperWASP,][]{norton2011short}, ASAS \citep[ASAS,][]{pojmanski2000all}, \citep[ROTSE,][]{akerlof2000rotse},  }
\label{fig: modern_surveys}
\end{figure}

%Arrange chronologically?

\subsection[The Kepler Spacecraft]{\hyperlink{toc}{The Kepler Spacecraft}}

\subsection[The Sloan Digital Sky Survey]{\hyperlink{toc}{The Sloan Digital Sky Survey}}

\cite{york2000sloan}

\citep{ivezic2007sloan} 

calibrations allow for stellar temperatures to be derived \cite{fukugita2011characterization}

SEGUE is a program that collects stellar spectra.

\subsection[The SuperWASP Survey]{\hyperlink{toc}{The SuperWASP Survey}}

follow-up on 1-meter telescopes in South Africa \citep{koen2016multi}.

more follow up \citep{darwish2016orbital}


\subsection[Future Surveys]{\hyperlink{toc}{Future Surveys}}

\begin{figure}[H]
\centering
\includegraphics[scale = 0.25]{future_missions.png}
\caption{}
\label{fig: future_missions}
\end{figure}



\section[Analysis Techniques]{\hyperlink{toc}{Analysis Techniques}} \label{sec: analysis_techniques}

\emph{Q: How do astronomers use their observations to learn about the physical characteristics of contact binary systems?}

Observational tests of theories of contact binaries \citet{lucy1979observational}

\subsection[Physical Light-Curve Modeling]{\hyperlink{toc}{Physical Light-Curve Modeling}} \label{sec: Physical Light-Curve Modeling}

In \S\ref{sec: Photometry of Contact Binaries}, we learned about how contact binary light-curves contain information about the physical nature of the system. Since the majority of data on contact binaries is in the form of photometric light-curve measurements, there has been much effort spent on refining the process of light-curve analysis.

In this section, we will learn how light-curves can be synthesized from physical models like those of \citet{lucy1968contact}. We also will learn how these synthesized light-curves can be fit to observations. It is important to know about the degeneracies present in this type of analysis. 

A large number of papers 

The Wilson-Devinney Code (hereafter WD code) \index{Wilson-Devinney Code} was the first code that could produce contact binary light-curves in large quantities. 

Fully automated approaches \citep{prsa2009fully} \citep{prsa2008artificial}

recent advances in modeling code \citep{prvsa2013physics}

useful review \citep{gimenez2006close}

\texttt{phoebe} is a user friendly implementation of the WD code in \texttt{python} developed by an international team of researchers. A researcher can feed a light-curve into \texttt{phoebe}. Assumptions about the model atmospheres and other aspects of stellar physics allow for a physical model to be fit to the data.

Almost always, \texttt{phoebe} models based on observational data are underdetermined. There are so many free parameters in the models (which allow the user to fit for starspots of various shapes and sizes, and include the light of an unresolved tertiary component), many degenerate perfect fits to observed data can always be found. Unfortunately, the WD code is frequently treated as a black box, and single-band photometry is used, parameters for spots and third light are fit. The degeneracy of these solutions is rarely if ever mentioned. 

However, radial velocity data can can be used to break degeneracies by providing an independent measurement of the system's total mass and mass-ratio \index{mass ratio, $q$}

Alternatives include \texttt{ROCHE} \index{\texttt{ROCHE} code} \citep{pribulla2012roche}

Simultaneous fitting of physical models of the contact binary light-curve and radial velocity data can provide almost totally determined solutions. 

\subsection[Nomographic Light-Curve Solutions]{\hyperlink{toc}{Nomographic Light-Curve Solutions}} \label{sec: Nomographic Light-Curve Solutions}

As larger and larger numbers of contact binaries were discovered, astronomers saw the motivation for providing a less computationally-intensive way of determining system geometry. 

\citet{mochnacki1972model} First nomographic solution, inspired work by Rucinski. \citep{rucinski1973w}

In this method, the physical model of the contact binary is used in reverse (as compared to \S\ref{sec: Physical Light-Curve Modeling}). A grid of models is produced, covering the parameter space of system geometry $[f,q,i]$. Then, light-curves are generated for each model. The observed light-curve is then compared with a whole catalog of computed light-curves. 

The main advantage of this method is that the model degeneracy can be estimated. Often, (especially for systems with a low orbital inclination) many models will be supported by the observed data.

OGLE work has determined a criterion for overcontact based on Fourier coefficients \citep{rucinski1997eclipsing}. \citep{rucinski1993simple}

Rucinski has used eclipse half-widths and Fourier components to estimate geometrical properties of the system:

the reliability of the mass-ratio determination from light-curve only data \citep{hambalek2013reliability}

ROTSE study using fourier methods \citep{coker2013study}

phenomenological study \citep{andronov2012phenomenological}

employing neural networks to find true binary parameters \citep{zeraatgari2015neural}

Difference between $q_{sp}$ and $q_{ph}$ \citep{van1985contribution}

\citep{rucinski1973w} \citep{rucinski1993light} \citep{terrell2005photometric} \citep{hambalek2013}

\subsection[O-C Analysis]{\hyperlink{toc}{O-C Analysis}} \label{sec: O-C Analysis}

O-C Analysis is short for ``observed minus computed", referring to the fact that a measurement is the observed time of minimum light as compared to the computed time of minimum light. 

1. A full light curve is obtained for the system in question. \\
2. Based on this light curve, the period is calculated, and an epheremis is computed, listing all of the future times of minimum light. \\
3. Then, light curves are obtained at future epochs. The time of minimum light are compared to previous times of minimum light. The difference $(O-C)$ is plotted as a function of epoch. \\

On an O-C diagram, a linear change in period appears as a parabola.

\begin{figure}[H]
\centering
\includegraphics[scale = 0.25]{kalimeris1994orbital_4a.png}
\caption{ The O - C diagram of V566 Oph, fit by a least squares polynomial. Fig. 4a from \citet{kalimeris1994orbital}}
\label{fig: kalimeris1994orbital_4a}
\end{figure}

On orbital period change \citep{kalimeris1994orbital}

The orbital period of a contact binary can change for a variety of reasons, but an observed period change always implies underlying geometrical and structural changes.

In contact systems, orbital period changes can generally be divided into (1) short-term variations, which happen on decadal (10 year) timescales, and (2) long-term variations, which happen on a thermal timescale.

The long-term variations are caused by

1) Angular momentum loss due to magnetic braking.
2) mass loss through the $L_{2}$ point \index{$L_{2}$}
3) chemical evolution of the primary.

The short-term variations are caused by
1) The orbit of a third body
2) Redistribution of the angular momentum due to magnetic activity.
3) 
Is O-C Analysis stable against the appearance and disappearance of starspots and photometric noise? \citep{kalimeris2002starspots}

\begin{figure}[H]
\centering
\includegraphics[scale = 0.25]{kalimeris2002starspots_7.png}
\caption{ Fig. 7 from \citet{kalimeris2002starspots}}
\label{fig: kalimeris2002starspots_7}
\end{figure}

automated approach with SuperWASP \citep{lohr2015orbital}

In data from surveys, the data may be too sparse to estimate times of minimum light from multiple epochs within the survey. A Lomb-Scargle (or LS) \citep{scargle1982studies}

\citep{horne1986prescription} 

\begin{equation} \label{eqn: horne_baliunas}
|\delta P| = (0.01728\text{ s}) \times (\frac{N_{0}}{100})^{-\frac{1}{2}} \times (\frac{T_{eff}}{5\text{yr}})^{-1} \\  \times (\frac{A/ \sigma_{N}}{10})^{-1} \times (\frac{P}{0.2\text{d}})^{2}
\end{equation}


\citet{qian2001orbital} has observed orbital period changes which indicate TRO. 

\subsection[Doppler Imaging]{\hyperlink{toc}{Doppler Imaging}} \label{sec: doppler_imaging}

\section[Working with Survey Data]{\hyperlink{toc}{Working with Survey Data}} \label{sec: Working with Survey Data}

Data from large All-Sky surveys is very different in nature compared with data taken on a single night with a single telescope. Working with all-sky surveys presents huge advantages to working with traditional light-curve data, but it also has major drawbacks.

In ``traditional" variable star observing, an observer slews the telescope to the target at the beginning of the night, and then takes a continuous sequence of images (from which she will make photometric measurements) at regularly spaced time intervals, until the star has rotated one full period, or until morning twilight. The observe can only look at one target at once, but the selected target is observed many times in one night.

In All-Sky surveys, the observing mode is different. All throughout the night, the telescope pans to a field, taking a few images, and then rapidly moving on to the next field. A given source might only be observed one or two times in a given night. The survey operates night after night, and after several years, it has amassed hundreds of observations of any point on the sky.

When we look in a survey database for photometric measurements of a known contact binary we often see data that looks like the data in Figure \ref{fig: lc_unfolded}. 

\begin{figure}[H]
\centering
\includegraphics[width = \textwidth]{lc_unfolded.png}
\caption{Observations of a contact binary, as returned by CRTS. The x-axis is the time of observation (in days), and the y-axis is the relative flux of the observation. Vertical bars about each point denote the uncertainty in the relative flux measurement. Note that the observed flux varies significantly from observation to observation, but we cannot see the periodic nature of the variability with our eyes}
\label{fig: lc_unfolded}
\end{figure}

We know that the data in Figure \ref{fig: lc_unfolded} is not data from a source with constant brightness. Look at the size of the error bar on each point. The error bar on each point is much smaller than the scatter in the distribution. We would call this source a variable source.


Hidden in this data is an underlying periodic function - the light-curve caused by the rotation of the contact binary. The period is hidden in this data, we just need to find it.

The data that we have here is a \emph{time-series}\index{time-series}: a number of measurements of of the flux of a source at many different times. The problem of finding a period in time-series data is usually handled with a \emph{Fourier Transform} \index{Fourier Transform}.

In the Fourier Transform, 

The CRTS telescope does not return to this source at regular times.

The algorithm that is most widely used to find periods in unevenly sampled time-series \citet{scargle1982studies}

We calculate the phase of each observation, by dividing by the period found in the signal.

\begin{equation} \label{phase_fold}
\theta = \frac{(\text{Time } \% \text{ Period})}{\text{Period}}
\end{equation}

where (\%) is the ``modulo" (or remainder operator)

\begin{figure}[H]
\centering
\includegraphics[width = \textwidth]{lc_folded.png}
\caption{Observations of a contact binary by CRTS, folded by the period as detected by the lomb-scargle algorithm.}
\label{fig: lc_folded}
\end{figure}

In Figure \ref{fig: lc_folded}, we see survey data of a contact binary, after it has been folded by the orbital period. We see a coherent light-curve with a beautiful shape.

This phase-folded light-curve is similar to what would be obtained in a night of  observation by a single observer.

\section[Paper I: Introduction to the CRTS Data]{\hyperlink{toc}{Paper I: Introduction to the CRTS Data}} \label{sec: Paper I: Introduction to the CRTS Data}

This work was submitted to the Monthly Notices of the Royal Astronomical Society (MNRAS) on June 20th, 2016, and was accepted on August 19th, 2016.

\includepdf[pages = -]{Marshetal2016.pdf}

\section[Paper II: Geometrical Parameters]{\hyperlink{toc}{Paper II: Geometrical Parameters}} \label{sec: Paper II: Geometrical Parameters}

\subsection[Light-curve Features]{Light-curve Features} \label{sec: Light-curve Features}

In order to understand how the light-curves change as a function of temperature, we must first figure out a way to describe a light-curve in a way that makes sense, both to humans and computers. In data science, this problem is called ``feature selection" \index{feature selection}. Let's consider what our light curve data actually is: a set of measurements of the flux of a contact binary. These measurements are taken at a certain time ($t$), have a certain value, in our case a measurement of flux ($f$), and this measured value has an associated error ($e$). From CRTS, we obtain lots of these measurements. Our light curve looks like this:

\begin{table}[htdp]
\caption{Format of Raw Data from CRTS}
\begin{center}
\begin{tabular}{|c|c|c|} \hline
\textbf{time} ($t$) & \textbf{flux} ($f$) & \textbf{error} ($e$) \\ \hline
number & number & number \\ \hline
number & number & number \\ \hline
... & ... & ... \\
\end{tabular}
\end{center}
\label{default}
\end{table}%

In \S\ref{sec: pause}, we were able to construct a coherent light-curve out of survey data taken at random times by folding the data by the orbital period. Just as before, we can input out data into a Lomb-Scargle \index{Lomb-Scargle algorithm}, or similar period-finding algorithm, find the best period $p$ and fold the observations by that period. This returns us a new version of the light-curve:

\begin{table}[htdp]
\caption{Format of Phase-folded Data from CRTS}
\begin{center}
\begin{tabular}{|c|c|c|} \hline
\textbf{phase} ($\theta$) & \textbf{flux} ($f$) & \textbf{error} ($e$) \\ \hline
number & number & number \\ \hline
number & number & number \\ \hline
... & ... & ... \\
\end{tabular}
\end{center}
\label{default}
\end{table}%

In the eight years between 2005 and 2013, CRTS observes a given source roughly 350 times. In other words, it reports about 350 phases ($\theta$), 350 fluxes ($f$), and 350 errors ($e$) on those fluxes. So, the raw data comes to us as $\approx 350 \times 3 = 1050$ individual numbers. These numbers are perfectly valid descriptors of the light-curve, but they are not easily understandable, neither by a human nor a computer.

Thankfully, we can introduce some assumptions that will make the task of succinctly describing our light-curves easier. First, we assume that our light-curve is a \emph{continuous function}. This means that there are not jumps or breaks in the true variation of light. The light-curve has a value at every point in phase, and is differentiable at every point in phase. Second, we assume that \emph{the light-curve is periodic}: that the pattern of light variation will exactly repeat itself after some amount of time. In \S\ref{sec: dec_var}, we learn that this is not exactly true for contact binaries when we observe them over many years. But for now, this assumption will serve us well.

Now, armed with our two new assumptions and light-curve data, we can construct a light-curve function. There are many ways to construct a continuous, periodic function from a set of data. \\

1. Polynomial Fit. \\

\begin{multline} \label{eqn: polynomial_fit}
f(\theta) = a_{0} + a_{1}\theta + a_{1}\theta^{2} + a_{2}\theta^{3} + ... (0 \leq p < 1) \\
f(\theta) = \sum_{i = 0}^{n} a_{i}\theta^{i}  (0 \leq \theta < 1)
\end{multline}

2. Polynomial - Spline Fit. \\

\begin{multline} \label{eqn: spline_fit} 
f(p) = 
\end{multline}

\citep{gettel2006catalog} a spline fit is employed. \citep{akerlof1994application} explanation of spline fit.

3. Fourier (Harmonic) Fit. \\

\begin{multline} \label{eqn: harmonic_fit} 
f(p) =  \\
f(p) = \sum_{i = 0}^{n} a_{i} \cos(2 \pi i p)
\end{multline}

There are many other forms that can be used to represent a continuous periodic function - but these are the most obvious choices. Each has advantages and disadvantages. For example, if the light-curve has large derivatives at some points in the phase (it has ``sharp turns"), a polynomial spline fit may be the best choice.

We have elected to use the harmonic fit, because it has a history of use in the description contact binary light-curves. It is easy to implement and can fit the data accurately, provided that there are not sharp turns.

By choosing a fitting function, we have turned our raw data from CRTS (which was 1000 numbers) into a continuous, periodic function which we can use to derive other, physically meaningful features.

Recall (as in Eqn. \ref{eqn: light_curve}) 

\begin{figure}[H]
\centering
\includegraphics[scale = 0.25]{lc_features.png}
\caption{A graphical description of the light-curve features that we derive for contact binaries in our sample.}
\label{fig: lc_features}
\end{figure}

For each light-curve, we have derived the following geometrical features:

Areas (a1, a2, a3, a4). The features have units of energy, being products of phase (which is time), and flux ( which is rate at which energy is received by the telescope.) \\
Flux Differences (Amp, $\Delta$Min $\Delta$Max). These features have units of flux, because they are differences of fluxes.
Eclipse Width and Half Minimum. (h1width, h2width). These features have units of time, because they are differences of phases.

In the following mathematical descriptions of the light-curve features, we will compute the light-curve on a grid of at least 500 points spaced evenly in phase. The relative flux computed from the harmonic fit at a point $i$  is denoted $F_{i}$. This is the $y$-coordinate on the light-curve graph. The phase of the point $i$ is denoted by $P_{i}$. This is the $x$-coordinate on the light-curve graph.

The area features were computed as Reimann sums on a fixed, evenly spaced, grid of discrete phases:

\begin{multline} \label{area_features}
a1  = \frac{\sum_{i}F_{i}}{\sum_{i} 1} (0 < P_{i} \leq 0.25) \\
a2  = \frac{\sum_{i}F_{i}}{\sum_{i} 1}  (0.25 < P_{i} \leq 0.50) \\
a3  = \frac{\sum_{i}F_{i}}{\sum_{i} 1}  (0.50 < P_{i} \leq 0.75) \\
a4  = \frac{\sum_{i}F_{i}}{\sum_{i} 1}  (0.75 < P_{i} \leq 1.00) \\
\end{multline}

The following light-curve features have units of flux. The light-curve amplitude (Amp) is the total range of the flux variation, ranging from a maximum at 1.0 (by the definition of our flux, see Eqn.\ref{css_norm_flux}) to the lowest value of the light-curve, $F_{\text{min1}}$. The flux difference between eclipse minima ($\Delta$Min) is the difference between the flux at the second local minimum in the light curve $F_{\text{min2}}$, and the flux at the lowest value of the light-curve, $F_{\text{min1}}$. The flux difference between out-of-eclipse maxima 

\begin{multline} \label{flux_features}
\text{Amp} = 1.0 -  F_{\text{min1}} \\
\Delta \text{Min} =  F_{\text{min2}} - F_{\text{min1}} \\
\Delta \text{Max} = 1.0 - F_{\text{max}} \\
\end{multline}

We also compute two intermediate flux features that will help us calculate the eclipse full-width at half-minimum. 

\begin{multline}
F_{\text{h1}} = 0.5 + \frac{F_{\text{min1}}}{2} \\
F_{\text{h2}} = 0.5 + \frac{F_{\text{min2}}}{2} \\
\end{multline}

The following light-curve features have units of time. $P_{\text{h1}}$ and $P_{\text{h2}}$ are the closest points in phase.

\begin{multline} \label{time_features}
\text{h1width} = P_{\text{h1}} - P_{\text{h1}} \\
\text{h2width} = P_{\text{h2}} - P_{\text{h2}}  \\
\end{multline}


\cite{o1951so}

\section[Paper III: Optical and UV Variability]{\hyperlink{toc}{Paper III: Optical and UV Variability}} \label{sec: Paper III: Optical and UV Variability}

In our study, we aim to learn how the luminosity of contact binary systems vary on decadal timescales. In \S\ref{lc_morph} we talked about the light-curve as if it was one single function. In actuality, the shape of contact binary light-curves change over time (Eqn. \ref{light_curve_redux}). 

\begin{equation} \label{eqn: light_curve_redux}
f(\text{Phase, Time}) = \text{ Flux Received at Telescope}
\end{equation}

We can detect light-curve changes with a survey that observes the same set of contact binaries over a timespan of several years (like CRTS). In order to detect changes in the contact binary light-curve in CRTS data, we will re-use much of the machinery that we have developed in \S\ref{lc_morph}.

We detect deviations of the light-curve with respect to the harmonic fit performed on all CRTS observations. We can consider the harmonic the ``average" light-curve of the contact binary during the eight-year CRTS observation timespan. This is because the CRTS measurements are randomly, (but uniformly) sampled in time, and the harmonic fit assigns equal weight to every measurement.


\citep{bradstreet1988mapping}


\section[The Future]{\hyperlink{toc}{The Future}}

In this section, I'd like to outline a few projects that are possible with existing datasets. I will also summarize a few future surveys that I believe have great potential to improve our understanding of contact binary systems.

\subsection[Coronal Rotation with GALEX]{\hyperlink{toc}{Coronal Rotation with GALEX}}

$\pi$ steradian (quarter of the sky) imaging survey. GALEX also completed a deep imaging survey, where a small portion of the sky was imaged for a long time. The two (FUV, NUV) detectors on GALEX are \emph{photon counting}\index{photon counting}, meaning that they record the time of arrival (to 5 milliseconds) and location (x,y pixel coordinated) of each incident photon. This means that it is possible construct light-curves from this data.

The python package \texttt{gphoton}\index{gphoton} allows the user to easily access and download GALEX photon-counting data. The user can construct light-curves out of the individual photon counts, to perform a periodicity analysis.

\citet{mccale1996rosat} have shown that the coronal rotation rate of convective contact binaries can be measured by using X-ray light-curves.

\subsection[Period Changes with Evryscope]{\hyperlink{toc}{Period Changes with Evryscope}}

Evryscope is a wide-field, high-entendue survey instrument. It can perform photometry on the entire visible night-sky, down to magnitude $V \approx 16$. This places thousands of known contact binaries within its reach. What makes Evryscope special is its cadence: $V$-band photometry will be performed every 2-minutes on every star in the visible sky. The cadence of Evryscope enables complete light-curves to be constructed for  

\subsection[H$\alpha$ Fluxes in PTF]{\hyperlink{toc}{H$\alpha$ Fluxes in PTF}}

The Palomar Transient Facility

%------------------------------------------------

%------------------------------------------------

%----------------------------------------------------------------------------------------
% Conclusion
%----------------------------------------------------------------------------------------

\section[Conclusion]{\hyperlink{toc}{Conclusion}}

%----------------------------------------------------------------------------------------
% Appendix
%----------------------------------------------------------------------------------------

\appendix

%----------------------------------------------------------------------------------------
% BIBLIOGRAPHY
%----------------------------------------------------------------------------------------

\newpage
\bibliographystyle{plainnat}
\printindex
\bibliography{thesis_bib}


%----------------------------------------------------------------------------------------

\end{document}